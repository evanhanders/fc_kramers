%\documentclass[iop]{emulateapj}
\documentclass[aps, pre, onecolumn, nofootinbib, notitlepage, groupedaddress, amsfonts, amssymb, amsmath, longbibliography]{revtex4-1}
\usepackage{graphicx}
\usepackage{calrsfs}
%\usepackage{hyperref}
\usepackage{xcolor}
%\hypersetup{
%    colorlinks,
%    linkcolor={red!50!black},
%    citecolor={blue!50!black},
%    urlcolor={blue!80!black}
%}
\usepackage{bm}
\usepackage{natbib}
\usepackage{longtable}
\LTcapwidth=0.87\textwidth

\newcommand{\Div}[1]{\ensuremath{\nabla\cdot\left( #1\right)}}
\newcommand{\angles}[1]{\ensuremath{\left\langle #1 \right\rangle}}
\newcommand{\grad}{\ensuremath{\nabla}}
\newcommand{\RB}{Rayleigh-B\'{e}nard }
\newcommand{\stressT}{\ensuremath{\bm{\bar{\bar{\Pi}}}}}
\newcommand{\lilstressT}{\ensuremath{\bm{\bar{\bar{\sigma}}}}}
\newcommand{\nrho}{\ensuremath{n_{\rho}}}
\newcommand{\approptoinn}[2]{\mathrel{\vcenter{
	\offinterlineskip\halign{\hfil$##$\cr
	#1\propto\cr\noalign{\kern2pt}#1\sim\cr\noalign{\kern-2pt}}}}}

\newcommand{\appropto}{\mathpalette\approptoinn\relax}

%\newcommand\mnras{{MNRAS}}%

\begin{document}
\section{Something that works}
OK, so, I finally have something working.  Here's the run-down.

My initial conditions are an adiabatic polytrope,
$$
T = L_z + 1 - z, \qquad \rho = (L_z + 1 - z)^{m_{ad}},
$$
where $L_z$ is the depth of the atmosphere, $z = [0, L_z]$ is the
height, and $m_{ad} = 1/(\gamma-1)$ is the adiabatic polytropic index. For
now, I'll assume that the radiative conductivity doesn't vary with time, such that
$$
\kappa(z) = \kappa_0 * \rho^{-(1+a)} T^{(3-b)},
$$
where $a = 1$, $b = -7/2$, and the thermal diffusivity is 
$\chi = \kappa / (\rho c_P)$. This means that $\chi(z) = \chi_0 \rho^{-(2+a)} T^{(3 - b)}$, and that
$\chi_{bot}/\chi_{top} = e^{-n_\rho(2+a) + (3-b)n_\rho/m_{ad}}$. I create something
kind of like a Rayleigh or Grashof number,
$$
\mathcal{R} = \text{Ra Pr} \propto \frac{\epsilon^2}{\chi^2}
$$
I set $\chi$ such that $\kappa(z=0) = 1$ when $\epsilon = \text{Ra Pr} = 1$, or
$$
\chi_{top} = e^{(-a + (3-b)m_{ad})n_{\rho}}\frac{\epsilon^{\alpha}}{\sqrt{\text{Ra Pr}}}.
$$

In addition to the ``true'' thermal diffusivity, authors like \cite{kapyla&all2017}
use a subgrid-scale diffusivity to keep numerical stability in cases like these
where the conductivity can drop by orders of magnitude over the height of the
stratified atmosphere. To keep stability, there's an extra flux in the energy eqn
like
$$
F_{SGS} = -\kappa_{SGS}(\grad T - \grad T_{ad}) \sim -\kappa_{SGS}\grad S,
$$
where ``SGS'' stands for subgrid-scale, $\grad T_{ad} = \bm{g}/c_P$, and $S$
is the entropy gradient. From our previous work in \cite{anders&brown2017}, we
know that the entropy gradient and the Mach number are intricately linked, such
that
$$
\text{Ma}^2 \sim u^2 \sim |\grad S|.
$$

We assume that all of the flux at the lower boundary is carried by the true $\chi$,
and that all of the flux at the upper boundary is carried by $F_{SGS}$.  Under that
assumption, we find that 
$$
F_{bot} = -\kappa\grad T = \kappa_{bot}\frac{g}{c_P} =
F_{top} = -\kappa_{SGS} \grad S
$$
Doing some simple rearrangement, we find that
$$
\grad S \sim \frac{\kappa_{bot}}{\kappa_{SGS}}.
$$
If we want to achieve some scaling,
$$
\grad S \sim \epsilon^{\beta},
$$
and we know that
$$
\kappa_{SGS} \grad S = \kappa_b,
$$
then we must have 
$$
\kappa_{SGS} = \kappa_b \epsilon^{-\beta} = 
\frac{\epsilon^{\alpha-\beta}}{\sqrt{\text{Ra Pr}}}.
$$
Furthermore, we know that \emph{motions are driven at the upper surface}, and we
want the \emph{Reynolds number to not be a function of epsilon}. I'm going to make
a sketchy assumption that Kapyla made here.  Which is that
$$
\nu = \kappa_{SGS} \text{Pr},
$$
which is to say that $\nu$ is being set by a Prandtl number and the subgrid scale
flux at the top of the atmosphere.  With that being true, for an atmosphere whose
depth does not depend on Ra and Pr, we find that the Reynolds number has no scaling
so long as
$$
u \sim \nu,
$$
or
$$
\epsilon^{\beta/2} \sim \epsilon^{\alpha - \beta}\sqrt{\frac{\text{Pr}}{\text{Ra}}}.
$$
To remove the epsilon scaling, we find that
$$
\beta = \frac{2}{3}\alpha,
$$
or that
$$
\chi_{top} = e^{(-a + (3-b)m_{ad})n_{\rho}}\frac{\epsilon^{\alpha}}{\sqrt{\text{Ra Pr}}}.
$$
and
$$
\kappa_{SGS} = \frac{\epsilon^{\alpha/3}}{\sqrt{\text{Ra Pr}}}.
$$
Note also in this formulation that the buoyancy time becomes
$$
t_b = \sqrt{ L_z (c_p / g) / (\epsilon^{2\alpha/3})}, 
$$
as what matters is the size of entropy fluctuations, not the size of our
arbitrary epsilon parameter.

So far, this seems to work. For example, with $\alpha = 1$, and at Ra = $10^2$,
$n_\rho = 3$, aspect ratio 2, nx = nz = 128, and Pr = $1$, I find the values
in the following table. Note that these are run on my local machine, and that they
have some spectral ringing (actually a decent amount, it's pretty gross), 
so take these results with a grain of salt, but the
trend is what's important here.


\begin{center}
 \begin{tabular}{||c c c c||} 
 \hline
 $\epsilon$ & Re (avg/max) & Ma (avg/max) & time ($t_{buoy}$) \\ [0.5ex] 
 \hline\hline
 1 & 15.33/80.51 & 4.94e-2/2.62e-1 & 60 \\ 
 \hline
 $10^{-2}$ & 14.35/82.93 & 1.073e-2/8.664e-2 & 60 \\
 \hline
 $10^{-4}$ & 16.8/106.4 & 2.76e-3/2.27e-2 & 60 \\
 \hline
 $10^{-6}$ & 17.1/86.9 & 5.78e-4/2.96e-3 & 60 \\
 \hline
\end{tabular}
\end{center}

And at Ra = 10, I find


\begin{center}
 \begin{tabular}{||c c c c||} 
 \hline
 $\epsilon$ & Re (avg/max) & Ma (avg/max) & time ($t_{buoy}$) \\ [0.5ex] 
 \hline\hline
 1 & 7.07/23.01 & 6.4e-2/3.53e-1 & 60 \\ 
 \hline
 $10^{-2}$ & 6.55/33.39 & 1.54e-2/8.89e-2 & 60 \\
 \hline
 $10^{-4}$ & 6.26/36.6 & 3.28e-3/2.6e-2 & 60 \\
 \hline
 $10^{-6}$ & / & / & 60 \\
 \hline
\end{tabular}
\end{center}



\section{Constructing the dual-layer kramer's atmosphere}
This shouldn't be too hard. Basically, we have an atmosphere with a piecewise
temperature gradient,
$$
\grad T = \begin{cases}
\frac{\bm{g}}{(m_{\text{kram}} + 1)} & z <= L_{RZ} \\
\frac{\bm{g}}{c_P}                   & z > L_{RZ},
\end{cases}
$$
where $m_{\text{kram}} = (3-b)/(1+a)$, $L_{RZ}$ is set by $n_{\rho, RZ}$, and
it's not too hard of a setup. If we integrate that piecewise temperature gradient,
setting $T = 1$ at the top, and then set $\rho$ according to hydrostatic 
equilibrium with $\rho = 1$ at the top of the atmosphere, we'll have our 
thermodynamic profiles set. At that point, it's just a matter of choosing 
$\kappa_0$, $\kappa_{SGS}$, and $\nu$, which are set through Ra/Pr/$\epsilon$.
I think $\kappa_0$ should once again be set such that the flux entering the
bottom of the atmosphere = 1 when Ra = Pr = $\epsilon$ = 1, such that
$$
\bigg|-\kappa\grad T\bigg|_{bot} = \rho_{bot}^{1+a}T_{bot}^{-(3-b)},
$$
or, with $T = \rho = 1$ at the top of the atmosphere,
$$
\bigg| -\kappa\grad T\bigg|_{bot} 
=  e^{(n_{\rho, CZ} + n_{\rho, RZ})(1+a)} 
e^{-(3-b)(n_{\rho, CZ}/m_{ad} + n_{\rho, RZ}/m_{\text{kram}}}.
$$
If we set Ra and Pr at the bottom of the atmosphere, then if 
$$
\text{RaPr} = \frac{C\epsilon^{2\alpha}}{\chi_{bot}^2},
$$ 
and $\chi_{bot} = \kappa_{bot} / e^{(n_{\rho, CZ} + n_{\rho, RZ})} / c_P$,
then
$$
\bigg| -\kappa\grad T\bigg|_{bot}
= \sqrt{C} \frac{\epsilon^{\alpha}}{\sqrt{\text{RaPr}}}
e^{(n_{\rho, CZ} + n_{\rho, RZ})(2+a)} 
e^{-(3-b)(n_{\rho, CZ}/m_{ad} + n_{\rho, RZ}/m_{\text{kram}}} c_P
$$
If the flux is 1 when Ra = Pr = $\epsilon$ = 1, then
$$
C = e^{-2(n_{\rho, CZ} + n_{\rho, RZ})(2+a)}
e^{2(3-b)(n_{\rho, CZ}/m_{ad} + n_{\rho, RZ}/m_{\text{kram}}} / c_P
= \frac{\chi(z=0)}{\chi(z=L_z)c_P}
$$
Or,
$$
\text{RaPr} = \frac{[\chi(z=0)/\chi(z=L_z)] \epsilon^{2\alpha}}{\chi^2}.
$$
Then, $\kappa_{\text{SGS}}$ and $\nu$ are set, as before, at the top by
Pr and $\epsilon$.








\section{Work on understanding Kramers Opacity atmospheres (e.g., \cite{kapyla&all2017})}
In this work, we have a two-layer atmosphere: an unstable layer lying above a
stable layer. Let's basicallly try to make the exact same experiment as Kapyla did,
so let's assume that the atmosphere has polytropic stratification, or
\begin{equation}
T = \begin{cases}
T_b + \grad T_{RZ}z,&  z <= L_{RZ} \\
T_m + \grad T_{CZ}(z - L_{RZ}),& z > L_{RZ}
\end{cases}
,\qquad
\rho = \begin{cases}
\rho_b \left(1 + \frac{\grad T_{RZ}}{T_b}z\right)^{m_{RZ}},& z <= L_{RZ} \\
\rho_m \left(1 + \frac{\grad T_{CZ}}{T_m}(z - L_{RZ})\right)^{m_{CZ}},& z > L_{RZ}
\end{cases}.
\end{equation}
There's a lot of different variables in there, so let's break them down a bit.
$T_b$ and $T_m$ are the temperature values at the bottom ($b$) and match ($m$)
points in the atmosphere. In other words, they're the temperatures at the bottom
of the RZ and the bottom of the CZ, respectively. $\rho_b$ and $\rho_m$ are the
density analogues of those two. $L_{RZ}$ is the geometric depth of the radiative
zone. $\grad T_{RZ}$ and $\grad T_{CZ}$ are the temperature gradients of the two
polytropes. $m_{RZ}$ and $m_{CZ}$ are the polytropic indices of the two layers.
Thanks to polytropes having some really easy-to-understand properties, here's
some definitions on some of the variables we have kicking around:
\begin{equation}
\begin{split}
&T_m = T_b e^{-n_{\rho, RZ} / m_{RZ}} \\
&\rho_m = \rho_b e^{-n_{\rho, RZ}} \\
&L_{RZ} = \frac{T_b}{|\grad T_{RZ}|}\left(e^{-n_{\rho, RZ} / m_{RZ}} - 1\right) \\
&L_{CZ} = \frac{T_b}{|\grad T_{CZ}|}e^{-n_{\rho, RZ} / m_{RZ}}\left(e^{-n_{\rho, CZ} / m_{CZ}} - 1\right) \\
&\grad T_{CZ} = \frac{\bm{g}}{R(1 + m_{CZ})} \\
&\grad T_{RZ} = \frac{\bm{g}}{R(1 + m_{RZ})} \\
&\grad T_{ad} = \bm{g}/c_P.
\end{split}
\end{equation}
OK, so that takes care of a lot of things, and if we specify the number of density
scale heights in both parts of the atmosphere, the polytropic indicies, the
temperature and density at the bottom of the atmosphere, and the profile of
gravity as a function of height, then we end up with a fully specified stitched
polytrope.

The interesting new thing that makes setting up this problem a mess is the 
radiative conductivity profile,
\begin{equation}
\kappa(z) = \kappa_0 \rho^{-(1+a)}T^{3-b},
\label{eqn:kramers}
\end{equation}
where $a = 1$ and $b = -7/2$ for free-free interactions. Note that if we have
a constant gravity $\bm{g} = -\text{constant}\cdot\hat{z}$, then setting
$m = (3 - b) / (1 + a)$ will produce a stratification that has a constant
conductivity. This is a logical $m$ to set for the RZ, so it turns out that
$\kappa \grad T_{RZ}$ is constant in the RZ so long as gravity is a constant.
This means that if we can get this atmosphere functioning for a single polytrope
in the CZ, then we can easily attach on the stable layer down below.

\subsection{Figuring out the non-dimensionalization of a single unstable layer}
OK, so with the conductivity described in Eqn. (\ref{eqn:kramers}), we're going
to figure out a single, unstable layer that carries flux convectively, and in which
we can control the mach number and the Reynolds number through two separate control
knobs. If this atmosphere is polytropic, we have the following stratification:
\begin{equation}
T = T_{top}e^{n_\rho / m} + (\grad T)z, \qquad 
\rho = \rho_{top}e^{n_\rho}
\left[1 + \frac{\grad T}{T_{top}}e^{-n_\rho/m}z\right]^m
\end{equation}
And because
$$
\rho = \rho_{top}e^{n_\rho} T^m,
$$
the profile for the conductivity is
\begin{equation}
\kappa(z) = \kappa_0 \rho_{top}^{-(1+a)} e^{-(1+a)n_{\rho}} T^{3 - b - m(1+a)}
= K_0 T^{3 - b - m(1+a)},
\end{equation}
where $K_0 = \kappa_0\rho_{top}^{-(1+a)} e^{-(1+a)n_{\rho}}$ is a constant. The
change in the conductivity over the depth of the atmosphere is
$$
\frac{\kappa(0)}{\kappa(L_z)} = e^{(3-b)n_\rho/m - n_\rho(1+a)}.
$$
This is large, for example when $m = 1.5$, $b = -7/2$, $a = 1$, and $n_\rho = 3$, 
this fraction is O($10^3$). That means that only one part in $10^3$ of the flux
carried by conduction at the bottom of the atmosphere can be conducted at the top
of the atmosphere. Put simply, for a polytrope with constant $\grad T$,
\begin{equation}
-\kappa(z=0)\grad T = 
\left(1 - \frac{\kappa(z=L_z)}{\kappa(z=0)}\right)F_{\text{\text{conv, top}}}.
\end{equation}
For highly stratified atmospheres, the term that is ``1 - a ratio'' will be
roughly equal to 1. However, it is important to keep this term for atmospheres
without much stratification, or for atmospheres where $a$ and $b$ are different.

If we think about the polytropic analogue, there we had
$$
F_{cond} = -\kappa\grad T_0,
$$
where $\grad T_{ad} = -g/c_P = \grad T_0(-1 + \epsilon/c_P)$ (for $R = 1$), and so
$$
F_{cond, avail} = -\kappa(\grad T_0 - -\grad T_0(\epsilon/c_P - 1))
= -\kappa \grad T_0 \epsilon / c_P.
$$
So there, we found that the flux that convection had to carry was O($\epsilon$),
and that the mach number was O($\epsilon^{1/2}$). Thus, it's probably a safe
assumption for us to assume that
\begin{equation}
-\kappa(z=0)\grad T = 
\left(1 - \frac{\kappa(z=L_z)}{\kappa(z=0)}\right)F_c \epsilon,
\end{equation}
where $F_c$ is just some constant that we can use to make sure that we don't
encounter problems with machine precision, etc. I now define
$$
\mathcal{F} \equiv \left(1 - \frac{\kappa(z=L_z)}{\kappa(z=0)}\right),
$$
and also
$$
\chi_{top} \equiv \frac{K_0 T_{top}^{(3-b) - m(1+a)}}{\rho_{top} c_P}, \qquad
K_0 \equiv \frac{\chi_{top}\rho_{top}c_P}{T_{top}^{(3-b)-m(1+a)}},
$$
such that the flux balance becomes
$$
\frac{g}{R(1+m)}\chi_{top}\rho_{top}c_P e^{(n_\rho/m)\{(3-b)-m(1+a)\}}
= \mathcal{F} F_c \epsilon.
$$
Assuming that $\chi_{top}$ is set as
\begin{equation}
\chi_{top} = \sqrt{\frac{g L_z^3 \epsilon}{\text{Ra}\cdot\text{Pr}}},
\end{equation}
and where
$$
L_z = \frac{T_{top}}{\grad T} (e^{n_\rho/m} - 1).
$$
Solving for gravity, we find
\begin{equation}
g = \left[\frac{R(m+1) \mathcal{F} F_c}{c_p \rho_{top}}\right]^{2/3}
\frac{(\text{Ra Pr}\, \epsilon)^{1/3}}{L_z} 
e^{-\frac{2}{3}\frac{n_\rho}{m}(3 - b - m[1+a])}.
\end{equation}
This means that we have three things left to specify: $F_c$, $\rho_{top}$,
and $L_z$ (which is set by $T_{top}$).  Now we start making assumptions.

First, I think it's reasonable to set
$$
F_c \equiv (\text{Ra Pr})^{-1/2}.
$$
This means that the total flux in the system is 
$\mathcal{F} \epsilon / \sqrt{\text{Ra Pr}}$. 
This scales nicely as I would expect. If we find that the flux is too
small for numerical stability, we can always bump this up by a factor of 10, etc.
I will also arbitrarily set $T_{top} = \rho_{top} = 1$, as we have done in previous
systems.  In making these choices, I find that
\begin{equation}
g = \left[\frac{R(m+1) \mathcal{F}}{c_p}\right]^{2/3}
\frac{\epsilon^{1/3}\grad T}{e^{n_\rho/m} - 1 }
e^{-\frac{2}{3}\frac{n_\rho}{m}(3 - b - m[1+a])}.
\end{equation}
plugging in $\grad T = g / R(1+m)$, I realize I made some huge mistakes. $g$ falls
out.



\bibliography{./../../biblio}
\end{document}
