%\documentclass[iop]{emulateapj}
\documentclass[aps, pre, onecolumn, nofootinbib, notitlepage, groupedaddress, amsfonts, amssymb, amsmath, longbibliography]{revtex4-1}
\usepackage{graphicx}
%\usepackage{hyperref}
\usepackage{xcolor}
%\hypersetup{
%    colorlinks,
%    linkcolor={red!50!black},
%    citecolor={blue!50!black},
%    urlcolor={blue!80!black}
%}
\usepackage{bm}
\usepackage{natbib}
\usepackage{longtable}
\LTcapwidth=0.87\textwidth

\newcommand{\Div}[1]{\ensuremath{\nabla\cdot\left( #1\right)}}
\newcommand{\angles}[1]{\ensuremath{\left\langle #1 \right\rangle}}
\newcommand{\grad}{\ensuremath{\nabla}}
\newcommand{\RB}{Rayleigh-B\'{e}nard }
\newcommand{\stressT}{\ensuremath{\bm{\bar{\bar{\Pi}}}}}
\newcommand{\lilstressT}{\ensuremath{\bm{\bar{\bar{\sigma}}}}}
\newcommand{\nrho}{\ensuremath{n_{\rho}}}
\newcommand{\approptoinn}[2]{\mathrel{\vcenter{
	\offinterlineskip\halign{\hfil$##$\cr
	#1\propto\cr\noalign{\kern2pt}#1\sim\cr\noalign{\kern-2pt}}}}}

\newcommand{\appropto}{\mathpalette\approptoinn\relax}

%\newcommand\mnras{{MNRAS}}%

\begin{document}
\section{Work on understanding Kramers Opacity atmospheres (e.g., \cite{kapyla&all2017})}
In this work, we have a two-layer atmosphere: an unstable layer lying above a
stable layer. Let's basicallly try to make the exact same experiment as Kapyla did,
so let's assume that the atmosphere has polytropic stratification, or
\begin{equation}
T = \begin{cases}
T_b + \grad T_{RZ}z,&  z <= L_{RZ} \\
T_m + \grad T_{CZ}(z - L_{RZ}),& z > L_{RZ}
\end{cases}
,\qquad
\rho = \begin{cases}
\rho_b \left(1 + \frac{\grad T_{RZ}}{T_b}z\right)^{m_{RZ}},& z <= L_{RZ} \\
\rho_m \left(1 + \frac{\grad T_{CZ}}{T_m}(z - L_{RZ})\right)^{m_{CZ}},& z > L_{RZ}
\end{cases}.
\end{equation}
There's a lot of different variables in there, so let's break them down a bit.
$T_b$ and $T_m$ are the temperature values at the bottom ($b$) and match ($m$)
points in the atmosphere. In other words, they're the temperatures at the bottom
of the RZ and the bottom of the CZ, respectively. $\rho_b$ and $\rho_m$ are the
density analogues of those two. $L_{RZ}$ is the geometric depth of the radiative
zone. $\grad T_{RZ}$ and $\grad T_{CZ}$ are the temperature gradients of the two
polytropes. $m_{RZ}$ and $m_{CZ}$ are the polytropic indices of the two layers.
Thanks to polytropes having some really easy-to-understand properties, here's
some definitions on some of the variables we have kicking around:
\begin{equation}
\begin{split}
&T_m = T_b e^{-n_{\rho, RZ} / m_{RZ}} \\
&\rho_m = \rho_b e^{-n_{\rho, RZ}} \\
&L_{RZ} = \frac{T_b}{|\grad T_{RZ}|}\left(e^{-n_{\rho, RZ} / m_{RZ}} - 1\right) \\
&L_{CZ} = \frac{T_b}{|\grad T_{CZ}|}e^{-n_{\rho, RZ} / m_{RZ}}\left(e^{-n_{\rho, CZ} / m_{CZ}} - 1\right) \\
&\grad T_{CZ} = \frac{\bm{g}}{R(1 + m_{CZ})} \\
&\grad T_{RZ} = \frac{\bm{g}}{R(1 + m_{RZ})} \\
&\grad T_{ad} = \bm{g}/c_P.
\end{split}
\end{equation}
OK, so that takes care of a lot of things, and if we specify the number of density
scale heights in both parts of the atmosphere, the polytropic indicies, the
temperature and density at the bottom of the atmosphere, and the profile of
gravity as a function of height, then we end up with a fully specified stitched
polytrope.

The interesting new thing that makes setting up this problem a mess is the 
radiative conductivity profile,
\begin{equation}
\kappa(z) = \kappa_0 \rho^{-(1+a)}T^{3-b},
\end{equation}
where $a = 1$ and $b = -7/2$ for free-free interactions. Note that if we have
a constant gravity $\bm{g} = -\text{constant}\cdot\hat{z}$, then setting
$m = (3 - b) / (1 + a)$ will produce a stratification that has a constant
conductivity.


\bibliography{./../../biblio}
\end{document}
