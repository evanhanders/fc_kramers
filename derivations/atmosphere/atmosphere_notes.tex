%\documentclass[iop]{emulateapj}
\documentclass[aps, pre, onecolumn, nofootinbib, notitlepage, groupedaddress, amsfonts, amssymb, amsmath, longbibliography]{revtex4-1}
\usepackage{graphicx}
\usepackage{calrsfs}
%\usepackage{hyperref}
\usepackage{xcolor}
%\hypersetup{
%    colorlinks,
%    linkcolor={red!50!black},
%    citecolor={blue!50!black},
%    urlcolor={blue!80!black}
%}
\usepackage{bm}
\usepackage{natbib}
\usepackage{longtable}
\LTcapwidth=0.87\textwidth

\newcommand{\Div}[1]{\ensuremath{\nabla\cdot\left( #1\right)}}
\newcommand{\angles}[1]{\ensuremath{\left\langle #1 \right\rangle}}
\newcommand{\grad}{\ensuremath{\nabla}}
\newcommand{\RB}{Rayleigh-B\'{e}nard }
\newcommand{\stressT}{\ensuremath{\bm{\bar{\bar{\Pi}}}}}
\newcommand{\lilstressT}{\ensuremath{\bm{\bar{\bar{\sigma}}}}}
\newcommand{\nrho}{\ensuremath{n_{\rho}}}
\newcommand{\approptoinn}[2]{\mathrel{\vcenter{
	\offinterlineskip\halign{\hfil$##$\cr
	#1\propto\cr\noalign{\kern2pt}#1\sim\cr\noalign{\kern-2pt}}}}}

\newcommand{\appropto}{\mathpalette\approptoinn\relax}

%\newcommand\mnras{{MNRAS}}%

\begin{document}
\section{Work on understanding Kramers Opacity atmospheres (e.g., \cite{kapyla&all2017})}
In this work, we have a two-layer atmosphere: an unstable layer lying above a
stable layer. Let's basicallly try to make the exact same experiment as Kapyla did,
so let's assume that the atmosphere has polytropic stratification, or
\begin{equation}
T = \begin{cases}
T_b + \grad T_{RZ}z,&  z <= L_{RZ} \\
T_m + \grad T_{CZ}(z - L_{RZ}),& z > L_{RZ}
\end{cases}
,\qquad
\rho = \begin{cases}
\rho_b \left(1 + \frac{\grad T_{RZ}}{T_b}z\right)^{m_{RZ}},& z <= L_{RZ} \\
\rho_m \left(1 + \frac{\grad T_{CZ}}{T_m}(z - L_{RZ})\right)^{m_{CZ}},& z > L_{RZ}
\end{cases}.
\end{equation}
There's a lot of different variables in there, so let's break them down a bit.
$T_b$ and $T_m$ are the temperature values at the bottom ($b$) and match ($m$)
points in the atmosphere. In other words, they're the temperatures at the bottom
of the RZ and the bottom of the CZ, respectively. $\rho_b$ and $\rho_m$ are the
density analogues of those two. $L_{RZ}$ is the geometric depth of the radiative
zone. $\grad T_{RZ}$ and $\grad T_{CZ}$ are the temperature gradients of the two
polytropes. $m_{RZ}$ and $m_{CZ}$ are the polytropic indices of the two layers.
Thanks to polytropes having some really easy-to-understand properties, here's
some definitions on some of the variables we have kicking around:
\begin{equation}
\begin{split}
&T_m = T_b e^{-n_{\rho, RZ} / m_{RZ}} \\
&\rho_m = \rho_b e^{-n_{\rho, RZ}} \\
&L_{RZ} = \frac{T_b}{|\grad T_{RZ}|}\left(e^{-n_{\rho, RZ} / m_{RZ}} - 1\right) \\
&L_{CZ} = \frac{T_b}{|\grad T_{CZ}|}e^{-n_{\rho, RZ} / m_{RZ}}\left(e^{-n_{\rho, CZ} / m_{CZ}} - 1\right) \\
&\grad T_{CZ} = \frac{\bm{g}}{R(1 + m_{CZ})} \\
&\grad T_{RZ} = \frac{\bm{g}}{R(1 + m_{RZ})} \\
&\grad T_{ad} = \bm{g}/c_P.
\end{split}
\end{equation}
OK, so that takes care of a lot of things, and if we specify the number of density
scale heights in both parts of the atmosphere, the polytropic indicies, the
temperature and density at the bottom of the atmosphere, and the profile of
gravity as a function of height, then we end up with a fully specified stitched
polytrope.

The interesting new thing that makes setting up this problem a mess is the 
radiative conductivity profile,
\begin{equation}
\kappa(z) = \kappa_0 \rho^{-(1+a)}T^{3-b},
\label{eqn:kramers}
\end{equation}
where $a = 1$ and $b = -7/2$ for free-free interactions. Note that if we have
a constant gravity $\bm{g} = -\text{constant}\cdot\hat{z}$, then setting
$m = (3 - b) / (1 + a)$ will produce a stratification that has a constant
conductivity. This is a logical $m$ to set for the RZ, so it turns out that
$\kappa \grad T_{RZ}$ is constant in the RZ so long as gravity is a constant.
This means that if we can get this atmosphere functioning for a single polytrope
in the CZ, then we can easily attach on the stable layer down below.

\subsection{Figuring out the non-dimensionalization of a single unstable layer}
OK, so with the conductivity described in Eqn. (\ref{eqn:kramers}), we're going
to figure out a single, unstable layer that carries flux convectively, and in which
we can control the mach number and the Reynolds number through two separate control
knobs. If this atmosphere is polytropic, we have the following stratification:
\begin{equation}
T = T_{top}e^{n_\rho / m} + (\grad T)z, \qquad 
\rho = \rho_{top}e^{n_\rho}
\left[1 + \frac{\grad T}{T_{top}}e^{-n_\rho/m}z\right]^m
\end{equation}
And because
$$
\rho = \rho_{top}e^{n_\rho} T^m,
$$
the profile for the conductivity is
\begin{equation}
\kappa(z) = \kappa_0 \rho_{top}^{-(1+a)} e^{-(1+a)n_{\rho}} T^{3 - b - m(1+a)}
= K_0 T^{3 - b - m(1+a)},
\end{equation}
where $K_0 = \kappa_0\rho_{top}^{-(1+a)} e^{-(1+a)n_{\rho}}$ is a constant. The
change in the conductivity over the depth of the atmosphere is
$$
\frac{\kappa(0)}{\kappa(L_z)} = e^{(3-b)n_\rho/m - n_\rho(1+a)}.
$$
This is large, for example when $m = 1.5$, $b = -7/2$, $a = 1$, and $n_\rho = 3$, 
this fraction is O($10^3$). That means that only one part in $10^3$ of the flux
carried by conduction at the bottom of the atmosphere can be conducted at the top
of the atmosphere. Put simply, for a polytrope with constant $\grad T$,
\begin{equation}
-\kappa(z=0)\grad T = 
\left(1 - \frac{\kappa(z=L_z)}{\kappa(z=0)}\right)F_{\text{\text{conv, top}}}.
\end{equation}
For highly stratified atmospheres, the term that is ``1 - a ratio'' will be
roughly equal to 1. However, it is important to keep this term for atmospheres
without much stratification, or for atmospheres where $a$ and $b$ are different.

If we think about the polytropic analogue, there we had
$$
F_{cond} = -\kappa\grad T_0,
$$
where $\grad T_{ad} = -g/c_P = \grad T_0(-1 + \epsilon/c_P)$ (for $R = 1$), and so
$$
F_{cond, avail} = -\kappa(\grad T_0 - -\grad T_0(\epsilon/c_P - 1))
= -\kappa \grad T_0 \epsilon / c_P.
$$
So there, we found that the flux that convection had to carry was O($\epsilon$),
and that the mach number was O($\epsilon^{1/2}$). Thus, it's probably a safe
assumption for us to assume that
\begin{equation}
-\kappa(z=0)\grad T = 
\left(1 - \frac{\kappa(z=L_z)}{\kappa(z=0)}\right)F_c \epsilon,
\end{equation}
where $F_c$ is just some constant that we can use to make sure that we don't
encounter problems with machine precision, etc. I now define
$$
\mathcal{F} \equiv \left(1 - \frac{\kappa(z=L_z)}{\kappa(z=0)}\right),
$$
and also
$$
\chi_{top} \equiv \frac{K_0 T_{top}^{(3-b) - m(1+a)}}{\rho_{top} c_P}, \qquad
K_0 \equiv \frac{\chi_{top}\rho_{top}c_P}{T_{top}^{(3-b)-m(1+a)}},
$$
such that the flux balance becomes
$$
\frac{g}{R(1+m)}\chi_{top}\rho_{top}c_P e^{(n_\rho/m)\{(3-b)-m(1+a)\}}
= \mathcal{F} F_c \epsilon.
$$
Assuming that $\chi_{top}$ is set as
\begin{equation}
\chi_{top} = \sqrt{\frac{g L_z^3 \epsilon}{\text{Ra}\cdot\text{Pr}}},
\end{equation}
and where
$$
L_z = \frac{T_{top}}{\grad T} (e^{n_\rho/m} - 1).
$$
Solving for gravity, we find
\begin{equation}
g = \left[\frac{R(m+1) \mathcal{F} F_c}{c_p \rho_{top}}\right]^{2/3}
\frac{(\text{Ra Pr}\, \epsilon)^{1/3}}{L_z} 
e^{-\frac{2}{3}\frac{n_\rho}{m}(3 - b - m[1+a])}.
\end{equation}
This means that we have three things left to specify: $F_c$, $\rho_{top}$,
and $L_z$ (which is set by $T_{top}$).  Now we start making assumptions.

First, I think it's reasonable to set
$$
F_c \equiv (\text{Ra Pr})^{-1/2}.
$$
This means that the total flux in the system is 
$\mathcal{F} \epsilon / \sqrt{\text{Ra Pr}}$. 
This scales nicely as I would expect. If we find that the flux is too
small for numerical stability, we can always bump this up by a factor of 10, etc.
I will also arbitrarily set $T_{top} = \rho_{top} = 1$, as we have done in previous
systems.  In making these choices, I find that
\begin{equation}
g = \left[\frac{R(m+1) \mathcal{F}}{c_p}\right]^{2/3}
\frac{\epsilon^{1/3}\grad T}{e^{n_\rho/m} - 1 }
e^{-\frac{2}{3}\frac{n_\rho}{m}(3 - b - m[1+a])}.
\end{equation}
plugging in $\grad T = g / R(1+m)$, I realize I made some huge mistakes. $g$ falls
out.



\bibliography{./../../biblio}
\end{document}
