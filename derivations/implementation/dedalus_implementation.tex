%\documentclass[iop]{emulateapj}
\documentclass[aps, pre, onecolumn, nofootinbib, notitlepage, groupedaddress, amsfonts, amssymb, amsmath, longbibliography]{revtex4-1}
\usepackage{graphicx}
\usepackage{hyperref}
\usepackage{xcolor}
\hypersetup{
    colorlinks,
    linkcolor={red!50!black},
    citecolor={blue!50!black},
    urlcolor={blue!80!black}
}
\usepackage{bm}
\usepackage{natbib}
\usepackage{longtable}
\LTcapwidth=0.87\textwidth

\newcommand{\Div}[1]{\ensuremath{\nabla\cdot\left( #1\right)}}
\newcommand{\angles}[1]{\ensuremath{\left\langle #1 \right\rangle}}
\newcommand{\grad}{\ensuremath{\nabla}}
\newcommand{\RB}{Rayleigh-B\'{e}nard }
\newcommand{\stressT}{\ensuremath{\bm{\bar{\bar{\Pi}}}}}
\newcommand{\lilstressT}{\ensuremath{\bm{\bar{\bar{\sigma}}}}}
\newcommand{\nrho}{\ensuremath{n_{\rho}}}
\newcommand{\approptoinn}[2]{\mathrel{\vcenter{
	\offinterlineskip\halign{\hfil$##$\cr
	#1\propto\cr\noalign{\kern2pt}#1\sim\cr\noalign{\kern-2pt}}}}}

\newcommand{\appropto}{\mathpalette\approptoinn\relax}

\newcommand\mnras{{MNRAS}}%

\begin{document}
\section{Implementation of Kramer's Opacity in Dedalus (easier)}
After talking with Ben, I realize that the implementation I have below in
the next section is harder than it needs to be. Let's look at this a different way.
If we have a radiative conductivity of the form,
\begin{equation}
\kappa(z) = \kappa_0 \rho^{-(1 + a)} T^{3 - b},
\end{equation}
then we can taylor expand this conductivity into its constant, linear, and
nonlinear components,
\begin{equation}
\kappa(\rho, T) = \kappa(\rho_0, T_0) 
+ \frac{\partial \kappa}{\partial T}\bigg|_{\rho_0, T_0}(T - T_0)
+ \frac{\partial \kappa}{\partial \ln\rho}\bigg|_{\rho_0, T_0}(\ln\rho - \ln\rho_0)
+ \kappa_{\text{NL}}.
\end{equation}
To evaluate that, we need a couple of derivatives:
\begin{equation}
\begin{split}
&\frac{\partial \kappa}{\partial T}\bigg|_{\rho_0, T_0} = 
\kappa_0(3-b)\rho_0^{-(1+a)}T_0^{2-b} = (3-b)\frac{\kappa(\rho_0, T_0)}{T_0} \\
&\frac{\partial \kappa}{\partial \ln\rho}\bigg|_{\rho_0, T_0} =
\rho\frac{\partial \kappa}{\partial \rho}\bigg|_{\rho_0, T_0} =
-(1+a)\kappa_0 \rho_0^{-(1+a)} T_0^{3-b} = -(1+a)\kappa(\rho_0, T_0),
\end{split}
\end{equation}
and after quick substitions,
\begin{equation}
\kappa(\rho, T) = \kappa(\rho_0, T_0)\left( 1 +
(3 - b)\frac{T_1}{T_0} - (1 + a) \ln\rho_1 \right)
+ \kappa_{\text{NL}}
\end{equation}
At this point, I will define the constant, linear, and nonlinear parts of $\kappa$,
\begin{equation}
\begin{split}
&\kappa_\text{C} \equiv \kappa(\rho_0, T_0) \\
&\kappa_\text{L} \equiv \kappa_C\left( 1 +
(3 - b)\frac{T_1}{T_0} - (1 + a) \ln\rho_1 \right) \\
&\kappa_{\text{NL}} \equiv \kappa - \kappa_\text{C} - \kappa_{\text{L}}.
\end{split}
\end{equation}

On the LHS of the energy equation, we have a term
\begin{equation}
\text{DivFcond} = \Div{-\kappa\grad T} 
= -(\kappa\grad^2 T + \grad \kappa\cdot\grad T)
\end{equation}
This can be simply broken up into linear and nonlinear components,
\begin{equation}
\begin{split}
&\text{DivFcond\_L} = -\left(\kappa_\text{L}\grad^2 T_0 
+ \kappa_\text{C} \grad^2 T_1 
+ \grad T_0 \cdot \grad \kappa_\text{L}
+ \grad T_1 \cdot \grad \kappa_\text{C}\right) \\
&\text{DivFcond\_NL} = -\left(
(\kappa - \kappa_\text{L})\grad^2 T_0
+ (\kappa - \kappa_\text{C})\grad^2 T_1
+ \grad T_0 \cdot \grad(\kappa - \kappa_\text{L}) 
+ \grad T_1 \cdot \grad(\kappa - \kappa_\text{C}) \right)
\end{split}
\end{equation}



\section{Implementation of Kramer's Opacity in Dedalus (harder)}
\label{sec:intro}
Under the use of a Kramer's opacity, the radiative conductivity takes the form
\begin{equation}
\kappa(z) = \kappa_0 \rho^{-(1 + a)} T^{3 - b},
\end{equation}
where $\kappa_0$ is a constant (which is set by physics, and thus will be
set by our input parameters like Ra, $\epsilon$ in our runs), $\rho$
is the density profile, and $T$ is the temperature profile. For
conductive transport by free-free interactions, $a = 1$ and $b = -7/2$,
and that's what we'll be using in this letter. But for now let's keep
it general.

Our energy equation takes the form
\begin{equation}
\frac{D T}{d t} + (\gamma-1)T\Div{\bm{u}} 
 + \frac{1}{\rho c_V}\Div{-\kappa(z) \grad T} = \text{Viscous heating term},
\end{equation}
and while we had easy forms of $\kappa$ before, we are now dealing with a much
dicier form. So let's expand it out, pull out linear terms, and get it in a
form that we can implement in Dedalus.

It's important to note that $\kappa$ is a scalar, so
$$
\Div{-\kappa \grad T} = -\kappa \Div{\grad T} - \grad T \cdot \grad \kappa
= -\kappa \grad^2 T - \grad T \cdot \grad \kappa,
$$
as per the plasma formulary, identity (7). Before we can really move on, we have
to plug in $\kappa$. We get
$$
-\kappa \grad^2 T - \grad T \cdot \grad \kappa
= - \kappa_0 \left(\rho^{-(1 + a)} T^{3-b}\grad^2 T + \grad T \cdot \grad (\rho^{-(1 + a)} T^{3-b}) \right),
$$
and expanding that last term, we get that
$$
\frac{1}{\rho c_V}\Div{-\kappa(z) \grad T} =
-\frac{\kappa_0}{c_V}\left[ \rho^{-(2+a)}T^{3-b}\grad^2T 
- (1 + a)\rho^{-(2+a)}T^{3-b}\grad T\cdot \grad\ln\rho
+ (3 - b)\rho^{-(2+a)}T^{2-b}\grad T\cdot\grad T   \right]
$$
Or, as simplified as possible (at this point),
\begin{equation}
\frac{1}{\rho c_V}\Div{-\kappa(z) \grad T} =
-\frac{\kappa_0\rho^{-(2+a)}}{c_V}\left[
T^{3-b}\grad^2 T - (1 + a)T^{3-b} \grad T \cdot \grad \ln \rho
+ (3 - b) T^{2-b}|\grad T|^2
\right].
\end{equation}
Yuck.

So looking at that mess of a term there, we need to break out the linear parts
and the nonlinear parts so that we can solve the linear parts implicity to recover
as much sanity and happiness as we can. The leading $\rho$ can be dealt with as
follows:
$$
\rho^{-(2+a)} = \rho_0^{-(2+a)}\left(1 + \left[e^{-(2+a)\ln\rho_1} - 1\right]\right].
$$
This peels out the leading linear component of this term, to be used on the linear
side of the equation, but it magically adds and subtracts 1 so that we keep our
equation true. Note however that there's no linear component of this rho mess.
Because we're in $\ln\rho$ variables, there's only the constant leading term
that we're taking advantage of as well as nonlinear terms that we're sweeping to
the RHS. This means that we need to pick out all of the linear terms that come
from the rest of the mess that makes up the Kramer's stuff.

I want to note that, in general,
$$
(C_1 + C_2)^a = C_1^a + a C_1^{a-1}C_2 + \cdots,
$$
such that if we have, for example,
$$
(T_0 + T_1)^4 = T_0^4 + 4 T_0^3 T_1 + \cdots,
$$
we can pick a constant term and a linear term out of such a combination generally.
So in our opacity, we have these linear terms, as far as I can tell:
\begin{align*}
&T_0^{3-b}\grad^2 T_1 + (3-b)T_0^{2-b}T_1 \grad^2 T_0 \\
&- (1 + a)T_0^{3-b}( \grad T_0 \cdot \grad \ln \rho_1 + \grad T_1 \cdot \grad\ln\rho_0) - (1 + a)(3-b)T_0^{2-b}T_1 \grad T_0 \cdot\grad\ln\rho_0 \\
&+ (3 - b) T_0^{2-b}(2 \partial_z T_0 \partial_z T_1) + (3-b)(2-b)T_0^{1-b}T_1 (\partial_z T_0)^2,
\end{align*}
where I have assumed that the initial atmosphere has no stratification
horizontally, such that
$$
|\grad T|^2 = (\partial_z T_0)^2 + 2 \partial_z T_0 \partial_z T_1 +
(\partial_x T_1)^2 + (\partial_y T_1)^2 + (\partial_z T_1)^2,
$$
and there is one order(background) term, one linear term, and three nonlinear terms.


With this knowledge, I define the following substitutions:
\begin{align*}
& L1 = \grad^2 T_1 - (1 + a)(\grad T_0\cdot\grad\ln\rho_1 + \grad T_1\cdot\grad\ln\rho_0) \\
& L2 = \grad^2 T_0 - (1 + a)\grad T_0\cdot\grad\ln\rho_0 \\
& L3 = 2(3-b)\frac{\partial T_0}{\partial z}\frac{\partial T_1}{\partial z} \\
& L4 = (3-b) \left(\frac{\partial T_0}{\partial z}\right)^2,
\end{align*}
and then the linear component of the divergence of the conductive flux is:
\begin{equation}
\boxed{
\text{LinKramer} = -\frac{\kappa_0 \rho_0^{-(2+a)}}{c_V}\left[
T_0^{3-b}(L1) + (3-b)T_0^{2-b}T_1(L2) + T_0^{2-b}(L3) + (2-b)T_0^{1-b}T_1(L4)
\right]
}.
\end{equation}

The nonlinear component is a bit uglier, but...it's still manageable. Basically we
have to (1) account for all of the explicitly nonlinear terms and then (2) account
for the nonlinear components of all of the linear components. Turns out this isn't
too bad.
\begin{equation}
\begin{split}
\text{NonLinKramer} &= -\frac{\kappa_0 \rho^{-(2+a)}}{c_V}\left[
-(1+a)T^3 \grad T_1\cdot\grad\ln\rho_1 + (3-b)T^{2-b}\left\{
\left(\frac{\partial T_1}{\partial x}\right)^2
+\left(\frac{\partial T_1}{\partial y}\right)^2
+\left(\frac{\partial T_1}{\partial z}\right)^2
\right\}\right]\\
&+ \left(e^{-(2+a)\ln\rho_1} - 1\right)(\text{LinKramer}) \\
&- \frac{\kappa_0 \rho^{-(2+a)}}{c_V}\left[
(T^{3-b} - T_0^{3-b})(L1) + (T^{3-b} - [3-b]T_0^{2-b}T_1)(L2)\right.\\
&\qquad\qquad\qquad\left.
+ (T^{2-b} - T_0^{2-b})(L3) + (T^{2-b} - [2-b]T_0^{1-b}T_1)(L4)
\right].
\end{split}
\end{equation}
Phew. Ok, so the first line of that nonlnear stuff is just all of the explicitly
nonlinear components. The second line, when added to LinKramer, converts the
$\rho_0$ in front to the full density, and then the following two lines are cleanup
to convert the linear or background $T$ exponentials to their full $T$ glory on
the RHS.  So if we go back to our initial equation, and we set it up as so:
\begin{equation}
\frac{D T}{d t} + (\gamma-1)T\Div{\bm{u}} 
 + \text{LinKramer} = \text{Viscous heating term} - \text{NonLinKramer},
\end{equation}
then it should properly pick out as much linear stuff as possible to our linear
side of the equation!


\bibliography{../biblio.bib}
\end{document}
