\documentclass[twocolumn, amsmath, amsfonts, amssymb]{aastex62}
\usepackage{mathtools}
\usepackage{bm}
\newcommand{\vdag}{(v)^\dagger}
\newcommand\aastex{AAS\TeX}
\newcommand\latex{La\TeX}

\newcommand{\Div}[1]{\ensuremath{\nabla\cdot\left( #1\right)}}
\newcommand{\DivU}{\ensuremath{\nabla\cdot\bm{u}}}
\newcommand{\angles}[1]{\ensuremath{\left\langle #1 \right\rangle}}
\newcommand{\KS}[1]{\ensuremath{\text{KS}(#1)}}
\newcommand{\KSstat}[1]{\ensuremath{\overline{\text{KS}(#1)}}}
\newcommand{\grad}{\ensuremath{\nabla}}
\newcommand{\RB}{Rayleigh-B\'{e}nard }
\newcommand{\stressT}{\ensuremath{\bm{\bar{\bar{\Pi}}}}}
\newcommand{\lilstressT}{\ensuremath{\bm{\bar{\bar{\sigma}}}}}
\newcommand{\nrho}{\ensuremath{n_{\rho}}}
\newcommand{\approptoinn}[2]{\mathrel{\vcenter{
	\offinterlineskip\halign{\hfil$##$\cr
	#1\propto\cr\noalign{\kern2pt}#1\sim\cr\noalign{\kern-2pt}}}}}

\newcommand{\appropto}{\mathpalette\approptoinn\relax}



%% Tells LaTeX to search for image files in the 
%% current directory as well as in the figures/ folder.
\graphicspath{{./}{figs/}}


\received{\today}
\revised{\today}
\accepted{\today}
\submitjournal{ApJ}

\shorttitle{Small name}
\shortauthors{Anders, Brown, and Rast}

\begin{document}

\title{Snappy and to the point.}

\correspondingauthor{Evan H. Anders}
\email{evan.anders@colorado.edu}

\author{Evan H. Anders}
\affil{University of Colorado -- Boulder}
\author{Benjamin P. Brown}
\affil{University of Colorado -- Boulder}
\author{Mark P. Rast?}
\affil{University of Colorado -- Boulder}


\begin{abstract}
\end{abstract}

\keywords{convection --- happy caterpillars}

\section{Introduction} \label{sec:intro}
This section probably needs to be written after first results start happening.
%What is the solar convective conundrum, why is it a problem, and what are people
%doing to combat it?
%
%Studies of Kramer's opacities and entropy rain are one path that people are using
%to figure out what's happening. Some pioneering work on that has been done
%recently, and they've found blah.
%
%In this paper, we study hydrodynamic convection in the optically thick
%limit such that a Newton-like conductivity law with a radiative conductivity
%is valid. For our opacity, we use a Kramer's-like opacity law of the form
%
%
%Which has been used recently by \cite{kapyla&all2017, kapyla&all2018}. We are particularly interested in studying
%the effects of this fully nonlinear condcutivity and its feedback on convective
%flows. In this work, we fix $a = 1$ and
%vary $b = (0, -3.5]$, where in the limit of $b = -3.5$, this radiative conductivity
%takes the same form as that for free-free interactions
%\citep{Cox&Giuli}. We take this approach,
%as we find that $b$ naturally controls the Mach number when the initial conditions
%are an adiabatic, hydrostatic polytrope. Regardless of the value of $b$, the
%radiative conductivity is highly nonlinear in both $T$ and $\rho$, and so we
%can study the importance of its nonlinear nature when the flows are high and
%low Mach number. The transition from $b = 0$ to $b = -3.5$ is also interesting
%in that it provides a gradual transition from classic polytropic convective
%systems \citep{hurlburt&all1984, brandenburg&all2005, anders&brown2017}, in
%which there is a large condition background conductive flux in the system,
%to systems in which the radiative flux becomes extremely inefficient in the
%interior and convection is required for nearly all energy transportation
%in the system.
%
%The importance of nonlinear conductive feedback has previously been studied in
%the context of mantle convection in the infinite Prandtl number limit
%\citep{dubuffet&all2000}, but there the conductivity is weakly inversely proportional
%to the temperature, whereas here it is strongly proportional to. Thus, we anticipate
%the negative feedback effects seen there to not be seen here. Blahblah.

We aim to answer the following questions:
\begin{enumerate}
\item How does the Mach number affect the importance of the nonlinearity? 
(low-hanging fruit)
\item How does average Plume penetration vary in 2D vs 3D? High Ma vs. low Ma?
\item Is there anything we can say about power spectrum / convective conundrum?
\end{enumerate}





\section{Experiment} \label{sec:experiment}
\subsection{Fluid Description \& Equations of Motion}
\label{sec:equations}
We study an ideal gas whose equation of state is
$P = \mathcal{R}\rho T$, where $P$ is the pressure, $\rho$ is the density,
$T$ is the temperature, and $\mathcal{R}$ is the ideal gas constant. 
We assume that the gas is made up of monatomic
particles such that the adiabatic index is $\gamma = 5/3$ such that the specific
heats at constant pressure and volume are $c_P = 2.5\mathcal{R}$ and $c_V = 1.5\mathcal{R}$.

This gas is arranged in plane-parallel, polytropically
stratified atmospheres \citep{anders&brown2017},
\begin{gather}
\rho_0(z) = \rho_{t}(1 + L_z - z)^m, \\
T_0(z)    = T_{t}(1 + L_z - z),
\end{gather}
where $\rho_{t}$ and $T_{t}$ are the values of density and temperature
at the top of the atmosphere,
$L_z$ is the depth of the atmosphere, $m$ is the polytropic index,
and $z$ is the vertical coordinate which increases with height in the
range $z=[0,L_z]$.
We choose here to study buoyantly stable, adiabatically stratified
polytropes with $m = m_{\text{ad}} = 1/(\gamma - 1) = 1.5$.
The atmosphere does not vary in the horizontal direction, 
and our cartesian domain
spans $x, y = [-L_\perp/2, L_\perp/2]$.

We nondimensionalize by setting $T_t = \rho_t = \mathcal{R} = 1$
at the top of the atmosphere.
By this choice, one non-dimensional unit of length is
the inverse temperature gradient scale of the polytrope, and 
one non-dimensional time unit is the isothermal sound crossing time of
that unit length.

The fluid velocity ($\bm{u} = u\hat{x} + v\hat{y} + w\hat{z}$), temperature ($T$)
and log density ($\ln\rho$) evolve in time according to the fully
compressible Navier-Stokes equations,
\begin{gather}
\frac{D \ln\rho}{D t} + \Div{\bm{u}} = 0
	\label{eqn:density_equation}
\\
\frac{D \bm{u}}{D t}  =
-\grad T - T \grad\ln\rho + \bm{g} + \frac{1}{\rho}\Div{\stressT}, 
	\label{eqn:momentum_equation}
\\
\begin{gathered}
\frac{D T}{D t} + T(\gamma - 1)\Div{\bm{u}}
+ \frac{1}{\rho c_V}\Div{-\kappa \grad T} = \\
\hspace{4cm}\frac{1}{\rho c_V}(\stressT\cdot\grad)\cdot\bm{u} + Q
	\label{eqn:energy_equation}
\end{gathered}
\end{gather}
where $D/Dt \equiv \partial/\partial t + \bm{u}\cdot\grad$, and $Q$ is
defined in Eqn.~\ref{eqn:q_cool}. The
viscous stress tensor is defined as 
\begin{equation}
\Pi_{ij} \equiv -\mu \left(\frac{\partial u_i}{\partial x_j} + 
\frac{\partial u_j}{\partial x_i} - \frac{2}{3}\delta_{ij}\grad\cdot\bm{u}\right).
	\label{eqn:stress_tensor}
\end{equation}
We decompose thermodynamic variables into constant background atmospheric terms
and evolving fluctuations about those backgrounds, $T = T_0(z) + T_1(x,y,z,t)$ and
$\ln\rho = \ln\rho_0(z) + \ln\rho_1(x,y,z,t)$. 
The domain is horizontally periodic, and at the upper and lower boundary
we impose impenetrable, stress free, fixed temperature boundary conditions,
such that
\begin{equation}
w = \partial_z u = \partial_z v = T_1 = 0 \text{ at } z = \{0, L_z\}.
\end{equation}

\subsection{Diffusivities}
\label{sec:diffusivities}
We study an optically thick medium in which
Fourier's law of conductivity 
(Eqn.~\ref{eqn:energy_equation} and \cite{lecoanet&all2014}) accurately
describes radiative conductivity in our system, and we assume a Kramer's like
opacity such that our radiative conductivity profile is 
\citep{barekat&brandenburg2014, kapyla&all2017, kapyla&all2018}
\begin{equation}
\kappa = K_0 \rho^{-(1+a)}T^{3-b}.
\label{eqn:kramers}
\end{equation}
Deep in the solar convection zone, where free-free particle interactions dominate
the opacity, a radiative conductivity of this form accurately
describes the physics felt by the fluid there with the exponents $a = 1$ and
$b = -3.5$ \citep{Cox&Giuli}, and we adopt these exponents in this work. 
As has been noted by previous authors \citep{jones1976, edwards1990, barekat&brandenburg2014},
a polytropic stratification where
$$
m = m_{\text{kram}} = \frac{3-b}{1 + a} = 3.25
$$
provides a thermally equilibrated, hydrostatically stable solution with a 
constant $\kappa$. Our choice of an adiabatic stratification of
$m = 1.5$ results in a $\kappa$ profile whose form is
\begin{equation}
\kappa_0 = K_0 T_0^{-b}.
\end{equation}
The divergence of this $\kappa$ profile results in a sizeable internal heating
term, particularly deep in the domain, that can drastically change the
stratification of the simulation over short timescales. In natural systems such
as the Sun, however, any sort of atmospheric evolution from this term would
take place on much longer timescales than the dynamical timescales of convective
elements. As a result, we add an effective internal cooling term,
\begin{equation}
Q = \frac{1}{\rho c_V}\partial_z(-\kappa_0 \partial_z T_0),
\label{eqn:q_cool}
\end{equation}
which cancels out the internal heating of divergence of the conductivity. This
allows us to study the motion of a plume in an adiabatic atmosphere
subject to a Kramer's-like opacity, as in Eqn.~\ref{eqn:kramers}.

For simplicity, we assume that the dynamic viscosity ($\mu$) 
of the fluid is constant. To determine its value, we set
 the Prandtl number at the top of the atmosphere,
$\text{Pr}_t = \text{Pr}(z=L_z) = \mu c_P / K_0$. 
As $\kappa$ increases with depth, this choice
results in a low Pr deep in the domain.

\subsection{Thermals}
\label{sec:thermal}
On top of our adiabatic background state, we impose a pressure-neutral ``thermal,''
a spherical thermal perturbation of the form [CITE DANIEL]
\begin{gather}
\label{eqn:thermal_T}
T_1 =  \frac{A_0}{2}\left(1 + \mathcal{N}\right) \left[1 - \text{erf}\left(\frac{r - r_0}{\delta}\right)\right], \\
\ln\rho_1 = \ln\left(\frac{T_0}{T_0 + T_1}\right),
\end{gather}
where $r \equiv \sqrt{(x-x_p)^2 + (y-y_p)^2 + (z-z_p)^2}$ is the distance
from $(x_p, y_p, z_p)$, the center of the thermal, $\delta$ determines how
geometrically steep the edges of the thermal are, 
$A_0$ determines the characteristic magnitude of the perturbation, and
 $\mathcal{N}$ is noise. 
This choice of initial conditions
creates a characteristic \emph{entropy} perturbation 
that will accelerate the thermal in an upwards or downwards direction depending
on the sign of $A_0$. Values for the properties of thermals defined in
Eqn.~\ref{eqn:thermal_T} are specified in Table \ref{table:thermal_init}.

The magnitude of $A_0$ is chosen so that the characteristic entropy of the thermal
is $\epsilon$, which controls the Mach number of the thermal, 
as in \cite{anders&brown2017}.
$\mathcal{N}$ is a turbulent noise term whose spatial power spectrum falls
in the manner of a turbulent cascade, $k^{-5/3}$, and which is normalized to
have a magnitude of 0.1.


\begin{table}[t]
%\renewcommand{\thetable}{\arabic{table}}
\tablewidth{\columnwidth}
\centering
\caption{Assigned values for parameters in Eqn.~\ref{eqn:thermal_T} for cold
(downflow) thermals and hot (upflow) thermals.} \label{table:thermal_init}
\begin{tabular}{c c c}
\hline
\hline
Parameter & Downflow & Upflow \\
\hline
\decimals
$\delta$ & $r_0/4$ & $r_0/4$ \\
$x_p, y_p$ & 0 & 0 \\
$z_p$ &  $L_z - r_0/2$   &  $r_0/2$     \\
$r_0$ &  $H_\rho |_{z=L_z}/2$     & $H_\rho |_{z=0}/10$     \\
$A_0$ &  $-(e^{\epsilon/c_P}-1)$   &   $T_0 |_{z=0}(e^{\epsilon/c_P}-1)$    \\
\hline
\end{tabular}
\end{table}

\subsection{Control Parameters}
Under these assumptions, there are a number of control parameters for
our experiment: $K_0$, $\text{Pr}_t$, $\epsilon$, and $L_z$, $L_\perp$.
We set $L_\perp = 20r_0$, such that the domain spans 10 thermal-widths.
We desire a domain which spans four density scale neights ($n_\rho = 4$),
so we set $L_z = e^{n_\rho / m} - 1 \approx 13.4$.  We set $\text{Pr}_t = 1$,
such that the full domain is at or below a Pr of unity. After making these
choices, we are left with two free control parmeters. We define
\begin{equation}
\mathcal{R} =  \frac{\sqrt{\epsilon}}{\mu} = \frac{\sqrt{\epsilon}}{K_0 \text{Pr}_t c_P},
\end{equation}
which is related to the evolved Reynolds number of the flow. In this
work, we will study the dynamics of thermals at varying values of
$\epsilon$ and $\mathcal{R}$, which in turn control the Mach number and
turbulence of the flows.

\subsection{Numerical Methods}
\label{sec:numerics}
We utilize the 
Dedalus\footnote{\url{http://dedalus-project.org/}} 
pseudospectral framework \cite{burns&all2016} to time-evolve  
(\ref{eqn:continuity_eqn})-(\ref{eqn:energy_eqn}) 
using an implicit-explicit (IMEX), third-order, four-step 
Runge-Kutta timestepping scheme RK443 \cite{ascher&all1997}.  
We study 2D and 3D simulations, and in our 2D runs we set $v = \partial_y = 0$.
Variables are time-evolved on a dealiased Chebyshev (vertical)
and Fourier (horizontal, periodic) domain in which the
physical grid dimensions are 3/2 the size of the coefficient grid.  
By using IMEX timestepping, we implicitly step the 
stiff linear acoustic wave contribution and are able to
efficiently study flows at high ($\sim 1$) 
and low ($\sim 10^{-4}$) Ma. This IMEX approach has been successfully 
tested against a nonlinear benchmark  of the compressible 
Kelvin-Helmholtz instability \cite{Lecoanet_et_al_2016_KH}.

\section{Results} \label{sec:results}
Figures:
\begin{enumerate}
\item Pretty dynamics, what the plume looks like, etc.
\item Size of $\kappa/\angles{\kappa} - 1$ vs. ...Ma? Time? Both?
\item Persistence depth vs. $\mathcal{R}$, $\epsilon$.
\item 2D vs. 3D persistence?
\item Power spectra?
\end{enumerate}


\section{Conclusions} \label{sec:conclusions}


\bibliography{../biblio.bib}

\end{document}


