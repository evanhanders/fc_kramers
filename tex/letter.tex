\documentclass[twocolumn]{aastex62}

\newcommand{\vdag}{(v)^\dagger}
\newcommand\aastex{AAS\TeX}
\newcommand\latex{La\TeX}

%% Tells LaTeX to search for image files in the 
%% current directory as well as in the figures/ folder.
\graphicspath{{./}{figs/}}


\received{\today}
\revised{\today}
\accepted{\today}
\submitjournal{ApJ}

\shorttitle{Small name}
\shortauthors{Anders, Brown, and Rast}

\begin{document}

\title{Snappy and to the point.}

\correspondingauthor{Evan H. Anders}
\email{evan.anders@colorado.edu}

\author{Evan H. Anders}
\affil{University of Colorado -- Boulder}


\begin{abstract}
\end{abstract}

\keywords{convection --- happy caterpillars}

\section{Introduction} \label{sec:intro}
Here's some introductory words where we talk about (maybe?) the convective
conundrum, and recent efforts to try to model the conductivity of the solar
convection zone properly. We talk about how those efforts try to resolve the
convective conundrum by properly driving the convection at the proper scales
(e.g., towards the top of the domain?). We also mention that those studies
make little to no mention of the Mach number.

It has been hypothesized that convection in the Sun is at low Rossby number,
which is to say that the convective flow speeds are small compared to rotational 
influences. If this is the case, it is important to study low Mach number
convection in order to properly understand the convective dynamics deep in the
solar interior.  Low Mach number convection is inherently coupled with
small fluctuations compared to the background state, and so we hypothesize that
in low Ma convection, it is not important to use a fully nonlinear conductivity
profile. This will also likely affect the degree of overshoot into the stable
layer below, and the overshoot should become less as the Mach number decreases.


In this letter we study convection in the presence of a realistic radiative
conductivity profile. We study a Kramer's-like opacity for free-free radiative
interactions, just like the deep layers of the solar convection zone. We study
both a fully nonlinear conductivity and a conductivity which (is constant in time?)
(only depends on the mean stratification?). We study both of these effects at high
and low Mach number in order to determine the importance of nonlinear 
conductivities at low Mach number.

\section{Experiment} \label{sec:experiment}
The atmosphere initially consists of two polytropic layers. The upper layer
is adiabatically stratified, and the lower layer is stably stratified.
Here's the equations for the atmosphere we use:

Here's the nondimensionalization of our atmosphere:

Here's our conductivity profile, diffusivities, control knobs:

Here's our numerical methods:


\section{Results} \label{sec:results}
Here's a figure that shows how Reynolds number and Mach number
scale with epsilon / Ra:

Here's a figure that shows the evolved stratification at high Ma
(fully nonlinear vs. not) and the evolved stratification at low Ma
(fully nonlinear vs. not). 

Here's a figure that shows the importance of nonlinearities in
the conductivity at both high and low Ma.

Here's a figure that shows either overshoot at high vs low Ma, or
a pretty figure of dynamics. Probably that one.

\section{Conclusions} \label{sec:conclusions}
What have we learned about opacity? Where are nonlinearities important?
Where are they unimportant?


\bibliography{../biblio.bib}

\end{document}


