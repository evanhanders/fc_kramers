\documentclass[twocolumn]{aastex62}

\newcommand{\vdag}{(v)^\dagger}
\newcommand\aastex{AAS\TeX}
\newcommand\latex{La\TeX}

%% Tells LaTeX to search for image files in the 
%% current directory as well as in the figures/ folder.
\graphicspath{{./}{figs/}}


\received{\today}
\revised{\today}
\accepted{\today}
\submitjournal{ApJ}

\shorttitle{Small name}
\shortauthors{Anders, Brown, and Rast}

\begin{document}

\title{Snappy and to the point.}

\correspondingauthor{Evan H. Anders}
\email{evan.anders@colorado.edu}

\author{Evan H. Anders}
\affil{University of Colorado -- Boulder}


\begin{abstract}
\end{abstract}

\keywords{convection --- happy caterpillars}

\section{Introduction} \label{sec:intro}
What is the solar convective conundrum, why is it a problem, and what are people
doing to combat it?

Studies of Kramer's opacities and entropy rain are one path that people are using
to figure out what's happening. Some pioneering work on that has been done
recently, and they've found blah.

In this paper, we study hydrodynamic convection in the optically thick
limit such that a Newton-like conductivity law with a radiative conductivity
is valid. For our opacity, we use a Kramer's-like opacity law of the form


Which has been used recently by \cite{kapyla&all2017, kapyla&all2018}. We are particularly interested in studying
the effects of this fully nonlinear condcutivity and its feedback on convective
flows. In this work, we fix $a = 1$ and
vary $b = (0, -3.5]$, where in the limit of $b = -3.5$, this radiative conductivity
takes the same form as that for free-free interactions
\citep{Cox&Giuli}. We take this approach,
as we find that $b$ naturally controls the Mach number when the initial conditions
are an adiabatic, hydrostatic polytrope. Regardless of the value of $b$, the
radiative conductivity is highly nonlinear in both $T$ and $\rho$, and so we
can study the importance of its nonlinear nature when the flows are high and
low Mach number. The transition from $b = 0$ to $b = -3.5$ is also interesting
in that it provides a gradual transition from classic polytropic convective
systems \citep{hurlburt&all1984, brandenburg&all2005, anders&brown2017}, in
which there is a large condition background conductive flux in the system,
to systems in which the radiative flux becomes extremely inefficient in the
interior and convection is required for nearly all energy transportation
in the system.

The importance of nonlinear conductive feedback has previously been studied in
the context of mantle convection in the infinite Prandtl number limit
\citep{dubuffet&all2000}, but there the conductivity is weakly inversely proportional
to the temperature, whereas here it is strongly proportional to. Thus, we anticipate
the negative feedback effects seen there to not be seen here. Blahblah.





\section{Experiment} \label{sec:experiment}
We study atmospheres whose initial conditions are polytropic,
as we did previously in \cite{anders&brown2017}. For all studies
in this work, we study an adiabatic polytrope where the initial
temperature and density stratification are

Here's the nondimensionalization of our atmosphere:

Here's our conductivity profile, diffusivities, control knobs:

Here's our numerical methods:


\section{Results} \label{sec:results}
Here's a figure that shows how Reynolds number and Mach number
scale with epsilon / Ra:

Here's a figure that shows the evolved stratification at high Ma
(fully nonlinear vs. not) and the evolved stratification at low Ma
(fully nonlinear vs. not). 

Here's a figure that shows the importance of nonlinearities in
the conductivity at both high and low Ma.

Here's a figure that shows either overshoot at high vs low Ma, or
a pretty figure of dynamics. Probably that one.

\section{Conclusions} \label{sec:conclusions}
What have we learned about opacity? Where are nonlinearities important?
Where are they unimportant?


\bibliography{../biblio.bib}

\end{document}


