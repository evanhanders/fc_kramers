\documentclass[twocolumn, amsmath, amsfonts, amssymb]{aastex62}
\usepackage{mathtools}
\usepackage{bm}
\newcommand{\vdag}{(v)^\dagger}
\newcommand\aastex{AAS\TeX}
\newcommand\latex{La\TeX}

\newcommand{\Div}[1]{\ensuremath{\nabla\cdot\left( #1\right)}}
\newcommand{\DivU}{\ensuremath{\nabla\cdot\bm{u}}}
\newcommand{\angles}[1]{\ensuremath{\left\langle #1 \right\rangle}}
\newcommand{\KS}[1]{\ensuremath{\text{KS}(#1)}}
\newcommand{\KSstat}[1]{\ensuremath{\overline{\text{KS}(#1)}}}
\newcommand{\grad}{\ensuremath{\nabla}}
\newcommand{\RB}{Rayleigh-B\'{e}nard }
\newcommand{\stressT}{\ensuremath{\bm{\bar{\bar{\Pi}}}}}
\newcommand{\lilstressT}{\ensuremath{\bm{\bar{\bar{\sigma}}}}}
\newcommand{\nrho}{\ensuremath{n_{\rho}}}
\newcommand{\approptoinn}[2]{\mathrel{\vcenter{
	\offinterlineskip\halign{\hfil$##$\cr
	#1\propto\cr\noalign{\kern2pt}#1\sim\cr\noalign{\kern-2pt}}}}}

\newcommand{\appropto}{\mathpalette\approptoinn\relax}



%% Tells LaTeX to search for image files in the 
%% current directory as well as in the figures/ folder.
\graphicspath{{./}{figs/}}


\received{\today}
\revised{\today}
\accepted{\today}
\submitjournal{ApJ}

\shorttitle{Small name}
\shortauthors{Anders, Brown, and Rast}

\begin{document}

\title{Snappy and to the point.}

\correspondingauthor{Evan H. Anders}
\email{evan.anders@colorado.edu}

\author{Evan H. Anders}
\affil{University of Colorado -- Boulder}


\begin{abstract}
\end{abstract}

\keywords{convection --- happy caterpillars}

\section{Introduction} \label{sec:intro}
What is the solar convective conundrum, why is it a problem, and what are people
doing to combat it?

Studies of Kramer's opacities and entropy rain are one path that people are using
to figure out what's happening. Some pioneering work on that has been done
recently, and they've found blah.

In this paper, we study hydrodynamic convection in the optically thick
limit such that a Newton-like conductivity law with a radiative conductivity
is valid. For our opacity, we use a Kramer's-like opacity law of the form


Which has been used recently by \cite{kapyla&all2017, kapyla&all2018}. We are particularly interested in studying
the effects of this fully nonlinear condcutivity and its feedback on convective
flows. In this work, we fix $a = 1$ and
vary $b = (0, -3.5]$, where in the limit of $b = -3.5$, this radiative conductivity
takes the same form as that for free-free interactions
\citep{Cox&Giuli}. We take this approach,
as we find that $b$ naturally controls the Mach number when the initial conditions
are an adiabatic, hydrostatic polytrope. Regardless of the value of $b$, the
radiative conductivity is highly nonlinear in both $T$ and $\rho$, and so we
can study the importance of its nonlinear nature when the flows are high and
low Mach number. The transition from $b = 0$ to $b = -3.5$ is also interesting
in that it provides a gradual transition from classic polytropic convective
systems \citep{hurlburt&all1984, brandenburg&all2005, anders&brown2017}, in
which there is a large condition background conductive flux in the system,
to systems in which the radiative flux becomes extremely inefficient in the
interior and convection is required for nearly all energy transportation
in the system.

The importance of nonlinear conductive feedback has previously been studied in
the context of mantle convection in the infinite Prandtl number limit
\citep{dubuffet&all2000}, but there the conductivity is weakly inversely proportional
to the temperature, whereas here it is strongly proportional to. Thus, we anticipate
the negative feedback effects seen there to not be seen here. Blahblah.





\section{Experiment} \label{sec:experiment}
\subsection{Model}
\label{sec:model}
We study an ideal gas whose equation of state is
$P = \mathcal{R}\rho T$. We assume that the gas is made up of monatomic
particles such that the adiabatic index is $\gamma = 5/3$ such that the specific
heats at constant pressure and volume are $c_P = 2.5\mathcal{R}$ and $c_V = 1.5\mathcal{R}$.

We study plane-parallel, polytropically
stratified atmosphere \citep{anders&brown2017},
\begin{gather}
\rho_0(z) = \rho_{t}(1 + L_z - z)^m, \\
T_0(z)    = T_{t}(1 + L_z - z),
\end{gather}
where $\rho$ and $T$ are the density and temperature, respectively,
$\rho_{t}$ and $T_{t}$ are their values at the top of the atmosphere,
$L_z$ is the depth of the atmosphere, $m$ is the polytropic index,
and $z$ is the vertical coordinate which increases with height in the
range $z=[0,L_z]$.
Throughout this work we will specifically study perfectly adiabatic
polytropes with $m = m_{\text{ad}} = 1/(\gamma - 1) = 1.5$.
The atmosphere does not vary horizontally, and our cartesian domain
spans $x = [-L_x/2, L_x/2]$ and $y = [-L_y/2, L_y/2]$.

We nondimensionalize our atmosphere by choosing
$\mathcal{R} = T_t = \rho_t = 1$.
By this choice, the non-dimensional
length scale is the inverse temperature gradient scale and the 
timescale is the isothermal sound crossing time, 
$\tau_I$, of this unit length.

On top of this reference atmosphere, we impose a ``thermal,''
a spherical thermal perturbation of the form [CITE DANIEL]
\begin{gather}
T_1(r) =  \frac{A_0}{2}\left(1 + \mathcal{N}\right) \left[1 - \text{erf}\left(\frac{r - r_0}{\delta}\right)\right], \\
\ln\rho_1 = \ln\left(\frac{T_0}{T_0 + T_1}\right),
\end{gather}
where $r \equiv \sqrt{(x-x_p)^2 + (y-y_p)^2 + (z-z_p)^2}$ is the distance
from the center of the perturbation, located at $(x_p, y_p, z_p)$, where
$$
r_0 =
\begin{cases}
H_\rho(z=L_z) / 2 & \text{downflow} \\
H_\rho(z=0) / 10  & \text{upflow},
\end{cases}
$$ 
is the radius of the thermal with $H_\rho$ the local density scale height, and 
$$
x_p = 0, y_p = 0, z_p = 
\begin{cases} 
L_z - \frac{r_0}{2} & \text{downflow} \\
\frac{r_0}{2} & \text{upflow}.
\end{cases}
$$
 $\delta = r_0 / 4$ sets the sharpness of the erf, and the erf approaches a step
function as $\delta \rightarrow 0$. 
$\mathcal{N}$ is symmetry-breaking noise, and $A_0$ determines the characteristic
magnitude of the perturbation. Subscripts ``0'' refer to background conditions,
and subscripts ``1'' refer to fluctuations. This choice of initial conditions
creates a characteristic \emph{entropy} perturbation while being pressure-neutral,
such that the initial motion of the thermal will be caused by buoyant acceleration
rather than pressure equilibration.
We set the amplitude of the perturbation such that
\begin{equation}
A_0 = \begin{cases}
- (e^{\epsilon/c_P} - 1), & \text{downflow}\\
T_0(z=0)\cdot(e^{\epsilon/c_P} - 1), & \text{upflow},
\end{cases}
\end{equation}
and by these choices the specific entropy perturbation 
of an upflow located near the bottom of
the domain or a downflow located near the top of the domain will be $\epsilon$,
and will control the Mach number of the thermal, as in \cite{anders&brown2017}.
$\mathcal{N}$ is a turbulent noise term whose spatial power spectrum falls
in the manner of a turbulent cascade, $k^{-5/3}$, and which is normalized to
have a magnitude of 0.1.




\subsection{Equations of Motion}
\label{sec:equations}
This fluid evolves according to the fully
compressible Navier-Stokes equations of hydrodynamics:
\begin{gather}
\frac{D \ln\rho}{D t} + \Div{\bm{u}} = 0
	\label{eqn:density_equation}
\\
\frac{D \bm{u}}{D t}  =
-\grad T - T \grad\ln\rho + \bm{g} + \frac{1}{\rho}\Div{\stressT}, 
	\label{eqn:momentum_equation}
\\
\begin{gathered}
\frac{D T}{D t} + T(\gamma - 1)\Div{\bm{u}}
+ \frac{1}{\rho c_V}\Div{-\kappa \grad T} = \\
\hspace{4cm}\frac{1}{\rho c_V}(\stressT\cdot\grad)\cdot\bm{u} + Q
	\label{eqn:energy_equation}
\end{gathered}
\end{gather}
where $D/Dt \equiv \partial/\partial t + \bm{u}\cdot\grad$, and $Q$ is
defined in Sec.~\ref{sec:atmosphere}. The
viscous stress tensor is defined as 
\begin{equation}
\Pi_{ij} \equiv -\mu \left(\frac{\partial u_i}{\partial x_j} + 
\frac{\partial u_j}{\partial x_i} - \frac{2}{3}\delta_{ij}\grad\cdot\bm{u}\right),
	\label{eqn:stress_tensor}
\end{equation}
We impose impenetrable, stress free, fixed temperature boundary conditions at
the top and bottom of the domain, with 
$w = \partial_z u = T_1 = 0$ at $z = \{0, L_z\}$. 


We assume that Fourier's law of conductivity \citep{lecoanet&all2014} accurately
describes radiative conductivity in our system, and we assume a Kramer's like
opacity such that our radiative conductivity profile is 
\citep{barekat&brandenburg2014, kapyla&all2017, kapyla&all2018}
\begin{equation}
\kappa(z, t) = K_0 \rho^{-(1+a)}T^{3-b}.
\end{equation}
Deep in the solar convection zone, a radiative conductivity of this form accurately
describes the physics felt by the fluid there for the exponents $a = 1$ and
$b = -3.5$, which describe the opacity of free-free interactions 
\citep{Cox&Giuli}. In this work, we choose $a = 1$ and study $b = (0, -3.5]$.
As has been noted by previous authors \citep{jones1976, edwards1990, barekat&brandenburg2014},
a polytropic stratification where
$$
m = m_{\text{kram}} = \frac{3-b}{1 + a}
$$
provides a hydrostatic reference solution with a constant $\kappa$ while the
temperature gradient is constant. Our choice of $a = 1$ chooses
$m_{\text{kram}} = 3/2 - b/2  = m_{\text{ad}} - b/2$, such that as the magnitude
of $b$ grows away from zero, the divergence of the 
conductivity profile of our adiabatic polytropic reference state grows, and in
some ways $b$ begins to resemble the classic superadiabatic excess of polytropic
solutions \citep{graham1975, anders&brown2017}.



\subsection{Numerical Methods}
\label{sec:numerics}
We utilize the 
Dedalus\footnote{\url{http://dedalus-project.org/}} 
pseudospectral framework \cite{burns&all2016} to time-evolve  
(\ref{eqn:continuity_eqn})-(\ref{eqn:energy_eqn}) 
using an implicit-explicit (IMEX), third-order, four-step 
Runge-Kutta timestepping scheme RK443 \cite{ascher&all1997}.  
Thermodynamic variables are decomposed such that $T = T_0 + T_1$ and
$\ln\rho = (\ln\rho)_0 + (\ln\rho)_1$, 
and the velocity is $\bm{u} = w\bm{\hat{z}} + u\bm{\hat{x}} + v\bm{\hat{y}}$.
In our 2D runs, $v = 0$.
Subscript 0 variables, set by (\ref{eqn:polytrope}), 
have no time derivative and vary only in $z$.
Variables are time-evolved on a dealiased Chebyshev (vertical)
and Fourier (horizontal, periodic) domain in which the
physical grid dimensions are 3/2 the size of the coefficient grid.  
By using IMEX timestepping, we implicitly step the 
stiff linear acoustic wave contribution and are able to
efficiently study flows at high ($\sim 1$) 
and low ($\sim 10^{-4}$) Ma.  Our equations take the form
of the FC equations in \cite{lecoanet&all2014}, extended to include
$\nu$ and $\chi$ which vary with depth, and we follow the approach there.
This IMEX approach has been successfully 
tested against a nonlinear benchmark  of the compressible 
Kelvin-Helmholtz instability \cite{Lecoanet_et_al_2016_KH}.

\subsection{Reference atmosphere}
\label{sec:atmosphere}



$Q$ is defined like 


\subsection{Initial conditions}
We do blahblahblah at low |$b$|.

We do blahblahblah at high |$b$|.

On top of this initial stratification, $T_1$ is initially filled with
random white noise whose magnitude is infinitesimal
compared to $|b| T_0$.
We filter this noise spectrum in coefficient space, 
such that only the lower 25\% of the coefficients
have power. 




\section{Results} \label{sec:results}
Here's a figure that shows how Reynolds number and Mach number
scale with $b$ / Ra:

Here's a figure of pretty dynamics at high and low Mach number. Probably
a few panels: entropy deviations, kappa deviations, and vertical velocity.

Here's a figure that shows the evolved stratification (where is BZ/DZ .../oz?) 
at high and low Mach number (and high and low $b$). Shows the progression towards
more nonlinearity in $\kappa$ and the evolved stratification we get out. This
is part of the "deviating away from polytropic convection" storyline.

Here's a figure that shows the importance of nonlinearities in
the conductivity at both high and low Ma. Basically, time-averaged scalar
values of $\kappa/\angles{\kappa} - 1$, and probably a time averaged
absv profile of that. Can we say why we see the scaling we see with $b$?


\section{Conclusions} \label{sec:conclusions}
What have we learned about opacity? Where are nonlinearities important?
Where are they unimportant?


\bibliography{../biblio.bib}

\end{document}


